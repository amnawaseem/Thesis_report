\thispagestyle{empty}
\vspace*{1.0cm}

\begin{center}
    \textbf{Abstract}
\end{center}

\vspace*{0.5cm}

\noindent
Over the past decade, virtualization technologies have gone from server consolidation in corporate data centers to small forms of embedded platforms for providing system security, hardware isolation, resource management and I/O device emulation. With the modern computers providing different forms of I/O devices, there has been huge development in hypervisors' technology to provide efficiency in the usage of physical resources. Major hypervisor vendors in virtualization market have been continuously developing different techniques to dynamically allocate memory and use resources more efficiently. However, this dynamicity in memory allocation introduces difficulty in provabilty and verification of configured hypervisor used in safety critical applications. PHIDIAS (Provable Hypervisor with Integrated Deployment Information and Allocated Structures) is a statically configured hypervisor developed by Security in Telecommunications department of Elektrotechnik und Informatik faculty at Technischen Universit�t Berlin, is built around the concept of Principle of Staticity to ease provability and reduce runtime complexity along with reduced memory footprint by eliminating dynamic elements from the system. However, it does not have support of I/O device virtualization. In this thesis, Xen Hypervisor's I/O device framework has been ported to PHIDIAS using its existing mechanisms of static memory sharing and inter-guest communication. Working and testing of ported virtualized network devices have been shown. Results are gathered for comparing throughput and bandwidth of virtual network interfaces on native Xen and PHIDIAS setup. Results highlight the fact that PHIDIAS delivers better performance for packets without fragmentation by bypassing the hypervisor layer for handling Xen specific hypercalls. However,current approach in ported network drivers introduces overhead while manually copying fragmented packets since PHIDIAS has no hypercall implemented for mapping shared pages to guest domain space by modifying corresponding page table entries as compared to Xen.
\\
\\
\keywords{I/O virtualization, Hypervisor, Xen, Static configuration, ARM architecture}
