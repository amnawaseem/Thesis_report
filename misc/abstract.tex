\thispagestyle{empty}
\vspace*{1.0cm}

\begin{center}
    \textbf{Abstract}
\end{center}

\vspace*{0.5cm}

\noindent
Over the past decade, virtualization technologies have gone from server consolidation in corporate data centers to small forms of embedded platforms for providing system security, hardware isolation, resource management and I/O device emulation. With the modern computers providing different forms of I/O devices, there has been huge development in hypervisors' technology to provide efficiency in the use of physical resources. Major hypervisors in virtualization market have been continuously deploying different memory techniques to dynamically allocate memory and use resources more efficiently. However, this dynamicity in memory allocation introduces difficulty in provabilty and verification of configured hypervisor used in safety critical applications. PHIDIAS, a static hypervisor developed by Security in Telecommunications department of Elektrotechnik und Informatik faculty at Technischen Universit�t Berlin, is built around the concept of Principle of Staticity to ease provability and reduce runtime complexity along with memory footprint by eliminating dynamic elements from the system. However, it does not have support of I/O device emulation. In this thesis, Xen Hypervisor's I/O device framework has been ported to PHIDIAS using existing mechanisms of static memory sharing and inter-guest communications.
