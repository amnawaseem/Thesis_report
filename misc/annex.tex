\appendix

\chapter{Appendix A \label{gitlinux}}
\renewcommand\thesection{\Alph{section}}
\section{Patch for Linux 4.11\_rc2 \label{linux_patch}}
Patch for changes in Linux kernel 4.11\_rc2 used for current work is placed at \\
\url{https://gitlab.sec.t-labs.tu-berlin.de/a.waseem/Thesis/blob/master/patch\_linux_4.11_rc2}.
\\
\\
Config file used for building Linux executable image for Dom0 guest is placed at \\
\url{https://gitlab.sec.t-labs.tu-berlin.de/a.waseem/Thesis/blob/master/config_dom0_usb}.
\\
\\
Config file used for building Linux executable image for DomU guest is placed at \\
\url{https://gitlab.sec.t-labs.tu-berlin.de/a.waseem/Thesis/blob/master/config_dom1_usb}.
\\
\\
Command used to cross compile Linux image is given below:
\begin{lstlisting}[caption=Command to cross compile Linux image for ARMv8 target on x86\_64 host,label={linuxcommand}]
make ARCH=arm64 CROSS_COMPILE=aarch64-linux-gnu- Image
\end{lstlisting}


\section{DTS file for HiKey ARMv8 Board \label{dts}}
DTS file for HiKey ARMv8 target platform used is placed at \\
\url{https://gitlab.sec.t-labs.tu-berlin.de/a.waseem/Thesis/blob/master/hi6220new1.dts}.
\\
\\
Command used to compile dts file is given below:
\begin{lstlisting}[caption=Command to compile dts file into dtb,label={dtscommand}]
dtc -I dts -O dtb -o hi6220new1.dtb hi6220new1.dts
\end{lstlisting}

\section{Patch for PHIDIAS changes \label{phidiaschanges}}
Patch for changes in PHIDIAS, related to adding serial multiplexer support for two guests running on different physical cores (0 and 1), for current work is placed at \\
\\
\url{https://gitlab.sec.t-labs.tu-berlin.de/a.waseem/Thesis/blob/master/patch_phidias}.
\\
\\
Scenario file used for building image for two guests (Linux1 for Dom0 and Linux2 for DomU) on PHIDIAS is placed at \\
\url{https://gitlab.sec.t-labs.tu-berlin.de/a.waseem/Thesis/blob/master/scenario.xml}.

\section{Patch for Xenstore Tool changes \label{xenstorechanges}}
Patch for changes in Xenstore are placed at \\
\url{https://gitlab.sec.t-labs.tu-berlin.de/a.waseem/Thesis/blob/master/patch_xen}.
\\
\\
\section{initramfs source code \label{xenstorechanges}}
Initramfs image was used by PHIDIAS's configuration framework for building final executable image in our current work. The source of initramfs is placed at \\
\url{https://gitlab.sec.t-labs.tu-berlin.de/a.waseem/Thesis/tree/master/initrd}.
\\
\\
For every modification in initramfs files, we have to recreate image in gzip format.The changed size of resultant initramfs image should be then placed in hi6220new1.dts file in chosen node as shown in listing \ref{chosen}.
\\
\begin{lstlisting}[caption= Chosen node in dts file for specifying initial ramdisk size ,frame=single,style=base, label={chosen}]
	chosen {
	stdout-path = "serial3:115200n8";
	bootargs = "console=ttyAMA3 earlycon=pl011,mmio32,f7113000 root=/dev/ram0 cmdline_from_dt init=/bin/sh loglevel=8";
	linux,initrd-start = <0x0 0xa000000>;
@	linux,initrd-end = <0x0 0xAEA9B59>;  @
	};
}

\end{lstlisting}
For this, a script has been developed which recreates the initramfs image, pack all files into an final initramfs file in gzip format and calculates the size which should be specified in \textbf{linux,initrd-end} option of chosen node in used dts file. This script is placed at \\
\url{https://gitlab.sec.t-labs.tu-berlin.de/a.waseem/Thesis/blob/master/script_size.sh}.
\\
\\

\chapter{Appendix B : Running Xenstore and Enabling Virtual Network Interfaces on PHIDIAS \label{running_xenstore}}

After loading final image, built by PHIDIAS configuration framework, on HiKey ARMv8 target, following commands in Dom0 guest are used to mount Xen Filesystem and start Xenstore daemon:\\

\begin{lstlisting}[caption=Commands to mount xenfs and start Xenstore daemon,label={initcomms}]
mount -t xenfs xenfs /proc/xen
cp -r /usr/local/lib/* /lib
mkdir /var/run
touch /var/run/xenstored.pid
touch /var/lib/xenstored/tdb
chmod +x /usr/local/sbin/xenstored
./usr/local/sbin/xenstored --pid-file /var/run/xenstored.pid --priv-domid 1
\end{lstlisting}

The next step is to probe netback driver in Dom0 guest. For this, a script \textbf{script\_xenstore.sh} has been written and placed in initramfs file. It is run using the following command:

\begin{lstlisting}[caption= Running script to probe netback driver in Dom0]
#sh script_xenstore.sh
\end{lstlisting}

After successful running of the script, vif1.0 interface will be created in Dom0. For triggering the probe of netfront driver in DomU, we have to send an IPI to DomU guest. For this, a kernel driver is developed which is also placed in initramfs file under directory `trugger'. Following command inserts the module in Dom0 kernel which triggers the IPI to DomU:
\begin{lstlisting}[caption= Running driver to probe netfront in DomU]
trigger# insmod testinit.ko
\end{lstlisting}

After successful completion of above command, a virtual network device named eth0 will be created in DomU. Before using these virtual interfaces for communication between Dom0 and DomU, following commands are issued to enable the devices and set their addresses:
\begin{lstlisting}[caption= Enabling and setting addresses of virtual network interfaces]
On Dom0:
#ifconfig vif1.0 hw ether 82:80:48:BA:d1:30
#ifconfig vif1.0 10.0.0.1 netmask 255.255.255.0
#ip link set dev vif1.0 up

On DomU:
#ifconfig eth0 10.0.0.2 netmask 255.255.255.0
#ip link set dev eth0
\end{lstlisting}


\endinput
