\thispagestyle{empty}
\vspace*{1cm}

\begin{center}
    \textbf{Zusammenfassung}
\end{center}

\vspace*{0.5cm}

\justify

In den letzten zehn Jahren sind Virtualisierungstechnologien von der Serverkonsolidierung in Unternehmens-Datenzentren hin zu kleinen Formen von eingebetteten Plattformen f�r Systemsicherheit, Hardwareisolierung, Ressourcenverwaltung und E / A-Ger�teemulation gegangen. Mit den modernen Computern, die verschiedene Formen von I / O-Ger�ten bereitstellen, hat es eine enorme Entwicklung in der Hypervisor-Technologie gegeben, um eine Effizienz bei der Verwendung von physischen Ressourcen bereitzustellen. Gro�e Hypervisor-Anbieter im Virtualisierungsmarkt haben kontinuierlich verschiedene Techniken entwickelt, um Speicher dynamisch zuzuweisen und Ressourc-
en effizienter zu nutzen. Diese Dynamizit�t bei der Speicherzuweisung f�hrt jedoch zu Schwier-igkeiten bei der Verifizierung und Verifizierung des konfigurierten Hypervisors, der in sicherheitskritischen Anwendungen verwendet wird. PHIDIAS (Proviable Hypervisor mit Integrated Deployment Information und Allocated Structures) ist ein statisch konfigurierter Hypervisor, der von der Abteilung Sicherheit in der Telekommunikation der Fakult�t f�r Elektrotechnik und Informatik der Technischen Universit�t Berlin entwickelt wurde und das Prinzip der Statik zur Erleichterung der Beweisbarkeit und zur Reduzierung der Laufzeitkomplexit�t beinhaltet. mit reduziertem Speicherbedarf durch Eliminierung dynamischer Elemente aus dem System. Die E / A-Ger�tevirtualisierung wird jedoch nicht unterst�tzt. In dieser Arbeit wurde das I / O-Device-Framework von Xen Hypervisor unter Verwendung der vorhandenen Mechanismen der statischen Speicherfreigabe und der Inter-Guest-Kommunikation nach PHIDIAS portiert. Das Arbeiten und Testen von portierten virtualisierten Netzwerkger�ten wurde gezeigt. Die Ergebnisse werden gesammelt, um den Durchsatz und die Bandbreite virtueller Netzwerkschnittstellen bei nativem Xen- und PHIDIAS-Setup zu vergleichen. Die Ergebnisse unterstreichen die Tatsache, dass PHIDIAS eine bessere Leistung f�r Pakete ohne Fragmentierung liefert, indem die Hypervisor-Schicht zur Behandlung von Xen-spezifischen Hypercalls umgangen wird. Der aktuelle Ansatz bei portierten Netzwerktreibern f�hrt jedoch zu einem Overhead, w�hrend fragmentierte Pakete manuell kopiert werden, da PHIDIAS kein Hypercall implementiert hat, um gemeinsam genutzte Seiten in den Gastdom�nenbereich zu mappen, indem entsprechende Seitentabelleneintr�ge im Vergleich zu Xen modifiziert werden.
\\
\\
\keywords{I/O-Virtualisierung, Hypervisor, Xen, statische Konfiguration, ARM-Architektur}\\
