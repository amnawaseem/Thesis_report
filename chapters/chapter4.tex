\chapter{Experimental Setup \label{cha:chapter4}}
In this chapter, the basic requirements and experimental setup for porting Xen split driver model to PHIDIAS will be discussed. 

\section{Equipment for Porting\label{sec:xen}}
Basic equipment used in this thesis consisted of an ARM target platform and a Linux PC with required tools and Linux source code. The details of target platform and Linux kernel used are given below.

\subsection{Target Platform\label{sec:xen}}
HiKey (LeMaker version) \cite{hikey} was used as a target platform. Its main features are:
\begin{itemize}
	\item Kirin 620 SoC
	\item ARM Cortex-A53 Octa-core 64-bit up to 1.2GHz (ARM v8 instruction set)
	\item 2GB LPDDR3 DRAM
	\item On-board 8GB eMMC nand flash storage
	\item TI WL1835MOD 2.4 GHz Wireless card
\end{itemize}

\subsection{Linux kernel version\label{sec:xen}}
The Linux kernel used for running Xen on HiKey was 4.1.15 obtained from \cite{96boards_2016}. Linux kernel used for running PHIDIAS was linux-4.11-rc2 obtained from \cite{linux4_11}. 

\subsection{Xen version\label{sec:xen}}
Xen 4.7.0 was used for setting up reference framework for porting downloaded from website \cite{xengit}.

\subsection{PHIDIAS version\label{sec:xen}}
Table \ref{PHIDIAS_comp} describes commit ids for components of PHIDIAS and their git repositories used in current project.

\begin{table}[!htbp]
	\centering
	\begin{tabular}[t]{|C{3cm}|C{4cm}  |C{4cm}|}
		\hline
		PHIDIAS component & commit id & git repo \\
		\hline
		xml &  30d880a5b2b28c7032c & git@gitlab.sec.t-labs.tu-berlin.de:phidias\\
		& dce4b564496ab08f06a15 &/xml.git \\
		\hline
		core & 3a1771b3aa6cf2b789805 & git@gitlab.sec.t-labs.tu-berlin.de:phidias\\
		& d849e74e8ffbdc9dbc3 & /serial\_multiplexer.git \\
		\hline
		abi &  8d8c3d1251799701eb& git@gitlab.sec.t-labs.tu-berlin.de:phidias/abi.git\\
		& 82e5a54fb7ec8075a30ce & \\
		\hline
		serial\_multiplexer & 968b913fa865e7129a9 & git@gitlab.sec.t-labs.tu-berlin.de:phidias \\
		& 10c79818153b361cbaf6e & /serial\_multiplexer.git\\
		\hline
	\end{tabular}
	\caption{Description of PHIDIAS components commit ids and git repos used for porting}
	\label{PHIDIAS_comp}
\end{table}

\section{Requirements for Porting\label{sec:xen}}
Following requirements were considered and fulfilled while porting Xen I/O split driver model to PHIDIAS:
\begin{itemize}
	\item Keep the changes as minimum as possible in PHIDIAS hypervisor code. For the current project, no additional changes are added in PHIDIAS besides adding serial multiplexer functionality for getting serial output from guests running on different physical CPUs. The link of patch for adding this functionality is provided in  section \ref{phidiaschanges} of Appendix A.
	\item Keep the changes as minimum as possible in Xen PV-enabled split drivers. General purpose interfaces should be implemented for the required changes, exposed to all Xen's split drivers. Current work had satisfied this requirement by adding changes usable by I/O split drivers.
	\item Add required changes in Linux kernel code cleanly which should be easy to maintain in future releases. Two major changes were added for porting work. One was to create a separate memory ZONE name ZONE\_XEN in Linux kernel. It was used by split drivers for allocating pages for sharing. Another change was to implement a mechanism for registering customized IPI handler for software generated interrupts.
	\item Port Xen's frontend and backend of network virtual device to PHIDIAS for the proof-of-concept of running it on a static hypervisor and adding networking support between guests.
	\item Same tools (Iperf and Ping) should be used for testing on Xen and PHIDIAS. Ping utility used is of BusyBox v1.22.1 and iperf is of version 2.0.10 (11 Aug 2017) pthreads.
	\item Keep changes small in Xenstore userspace tool. For current work, only one change is added to introduce domU with Xenstore manually. In original Xen setup, this is done via XL tool as explained previously \ref{sec:xenstore} which has dependencies on Xen hypervisor. Hence it was not used in our porting setup.
	\item At least two guests should be able to communicate via ported network split driver.
\end{itemize}

