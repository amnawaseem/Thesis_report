\chapter{Introduction\label{cha:chapter1}}

In this section, a brief introduction of virtualization and hypervisor technologies will be given. A little comparison of Xen, KVM and Phidias will be provided as well.

\section{Motivation\label{sec:moti}}

In this section, motivation behind adding device I/O emulation support in Phidias will be discussed e.g. to prove flexibility of a static hypervisor and to ease provability. This section will also highlights the reasons for choosing Xen as a reference.

\section{Objective\label{sec:objective}}

This section will describe the problems and issues which will be solved in this thesis. The main aim of this thesis will be given in this section e.g using existing Phidias memory sharing and IPI mechanism to port Xen Split Driver model for I/O virtualization and keep changes as minimal as possible.

\section{Scope\label{sec:scope}}

In this section, method for porting device I/O emulation of Xen to Phidias will be discussed i.e. main approach will be discussed to port split driver model to Phidias. 

\section{Outline\label{sec:outline}}

This section gives a brief introduction into the main chapters. 
\\
\\
\noindent This example thesis is separated into 7 chapters.
\\
\\
\textbf{Chapter \ref{cha:chapter2}} is called 'Background'. Here I will give background information to lay foundation of thesis topic.
\\
\textbf{Chapter \ref{cha:chapter3}} provides the requirements and assumptions of thesis
\\
\\
\textbf{Chapter \ref{cha:chapter4}} discussed  'Design' and 'Model'. Here I will describe my approach, give a high-level description to the architectural structure and to the basic components that my design consists of.
\\
\\
\textbf{Chapter \ref{cha:chapter5}} describes the implementation part of my work.
\\
\\
\textbf{Chapter \ref{cha:chapter6}} is 'Evaluation'. How the testing is performed for thesis will be discussed. Measurements, tests, screenshots will be included.
\\
\\
\textbf{Chapter \ref{cha:chapter7}} summarizes the thesis, describes the problems that occurred and gives an outlook about future work.