\chapter{Introduction \label{cha:chapter1}}

With a diverse range of applications and market opportunities of virtualization technology, it has become one of leading technologies of the future. Though virtualization began its journey from mainframes, it soon found its way into servers and desktop virtualization. By keeping its continuous improvements, it has now entered into today's most demanding market space of embedded systems.
\\
\\
A few decades ago, embedded systems were designed to run simpler applications with specific design constraints. But the present embedded systems are able to provide complex functionality, targeting real time applications and defense mission critical systems and support a diverse range of I/O devices. Providing optimal, flexible and robust hardware utilization of these complex I/O devices was a challenge faced by virtualization industry in recent years. For this purpose, efforts of virtualization industry has been focused on developing an efficient software, called hypervisor, that could create several different virtual machines hosting on a single platform and provide memory management, resource allocation and sharing of hardware devices in addition to the secure execution of those guest machines running on top of it. In short, we can say that running multiple mixed-criticality applications on embedded systems, providing intelligent I/O virtualization and ensuring their security were the major sources of evolution in virtualization technology. 
\\
\\
In the field of developing efficient software for better hardware utilization, multiple players are providing innovative solutions to meet the needs of virtualization industry. According to 2016 Spiceworks State of IT report \cite{spiceworks}, more than 76\% of organizations are taking advantage of virtualization today, and that number will continue to grow in the future. Xen and KVM are amongst the most popular open source hypervisors available in market. Both provide support for I/O device virtualization. However, they both reply upon dynamic memory and resource allocation techniques which makes it harder to prove their functional correctness which is a crucial requirement for the reliability of safety- and security-critical systems.
\\
\\
PHIDIAS is a hypervisor developed by Dr.-Ing. Jan Nordholz at TU Berlin Telekom Innovation Laboratories \cite{Jan},supporting ARM, MIPS and x86. It is based on static compile time configuration of virtual machines to remove non-mandatory dynamic components from the system. This static configuration feature of PHIDIAS reduces dynamicity from the system which results in a simpler provable code with a smaller memory footprint. However, it only provides virtualization of timer and UART devices. There is no support for other virtual I/O devices e.g. network and block devices.
\\
\\
Theoretically speaking, adding support of I/O virtualization in PHIDIAS could be done in two ways. One way is to write virtual device drivers for I/O devices from the scratch. And another method is to use I/O virtualization model of an already existing hypervisor and use it as a reference for porting those virtual device drivers to PHIDIAS.For our work, second method seems to be a reasonable approach as it would require less time. Now between Xen and KVM as two promising choices for using as a reference, Xen seems to be a better option. Most important reasons are that it is a thin type-1 or bare metal hypervisor with less areas of attacks and makes guests OSes responsible for implementing and maintaining virtual I/O drivers. On the other hand, KVM is a hosted or type-2 hypervisor which requires target processor to support hardware virtualization and uses VirtIO model \footnote{VirtIO is a virtualization standard for network and disk device drivers where just the guest's device driver "knows" it is running in a virtual environment, and cooperates with the hypervisor. This enables guests to get high performance network and disk operations, and gives most of the performance benefits of paravirtualization \cite{virtio}.} where backend drivers of PV I/O devices reside inside hypervisor \cite{kvm}.  A detailed view of Xen will be explained in this thesis for the better understanding of its I/O architecture.
\\
\\
This thesis will add support of I/O virtualization in PHIDIAS by using Xen I/O architecture as a reference. Due to time restrictions, it will focus on running and testing virtual network devices on PHIDIAS and compare network performance in terms of throughput and bandwidth. Methods for running other I/O devices will be discussed theoretically along with benefits and areas of future research.

\section{Motivation\label{sec:moti}}
There are two main motivational factors behind this work. First is to demonstrate the fact that even though PHIDIAS is a statically configured hypervisor with a minimal set of drivers, its structure is flexible enough to add support for other virtual I/O devices. And the second motivation is to compare the performance of virtualized network I/O devices between Xen and PHIDIAS.

\section{Aims and Objectives\label{sec:objective}}
Main aims of the current work are twofold. First is to investigate mechanisms in PHIDIAS for memory sharing and interprocessor communication and extending its feature set to accommodate ported virtual I/O devices. Second is to keep changes in Xen I/O drivers and PHIDIAS as minimum as possible in order to reuse maintenance support of Xen community and to keep the smaller footprint and static nature of PHIDIAS intact.
\\
\\
For achieving current aims, this thesis will focus on finding core building blocks of Xen I/O virtualization model and replace them with corresponding components of PHIDIAS. For running and testing the ported system, ARM will be used as a target platform due to its wide range of applications in smart phones, laptops, digital TVs, set-top boxes, mobile, and Internet of Things `IoT' devices. 

\section{Related Work\label{sec:prework}}
There has been a lot of work on developing techniques for high-performance I/O virtualization. Over the past decade, achieving better hardware utilization by decoupling logical I/O devices from physical ones has been the focus of researchers in virtualization domain.
\\
Gordon et al. \cite{exitless} proposed an exitless paravirtual I/O model, under which guests and the hypervisor, running on distinct cores, exchange exitless notifications instead of costly exit-based notifications for increasing the performance of paravirtual I/O. Nadav et al. \cite{efficient}  developed ELVIS (Efficient and scaLable paraVirtual I/O System) for running host functionality on dedicated cores that were separate from guest cores, and thus avoiding exits on the I/O path. Shafer et al. \cite{CDNA} introduced concurrent direct network access (CDNA), a new I/O virtualization technique combining both software and hardware components that significantly reduces the overhead of network virtualization in VMMs. With CDNA, a unique context is allocated to each virtual machine on the network interface for communicating directly with the network interface through that context. In this way, tasks of traffic multiplexing, interrupt delivery, and memory protection gets divided between hardware (network interfaces) and software (virtual  machine monitor). Shinagawa, Kawai and Kono et al. \cite{bitvisor} developed a small hypervisor named BitVisor for the fast I/O access from the guest operating systems by passing-through the virtual machine monitor.
\\
\\
In addition to developing innovating software techniques for efficient virtualization of I/O subsystems and peripheral devices, researchers have also worked on adding hardware support for virtualization in I/O devices. The idea is to bypass the hypervisor by virtual machines to directly access logical interfaces implemented on actual hardware, thus producing best performance. Himanshu and Karsten \cite{selfvirt} developed a new approach of I/O virtualization by offloading select virtualization functionality onto the device termed as self-virtualized devices. Intel's has developed Single Root I/O Virtualization (SR-IOV) technology \cite{SRIOV} that extends the physical function of the I/O port of an I/O device to a number of virtual functions directly assigned to virtual machines, hence achieving near native performance without software emulation. 
\\
\\
Broadly speaking, efforts for developing effective I/O virtualization techniques have resulted in three types of solutions i.e software-based, hardware-based and combination of both software and hardware based I/O virtualization. On one side, software-based approach, uses paravirtualized drivers implemented in dedicated driver domain or as a part of hypervisor and provides isolated execution of I/O virtual devices with live migration \cite{migration} and traffic monitoring \cite{monitoring} support for virtual machines. On the other side, hardware method reduces the software overhead and bypass hypervisor layer to achieve good performance. In the middle, there is a combination of both approaches. Research shows that all solutions have their pros and cons and the selection heavily depends upon the target application.

\section{Thesis structure \label{sec:outline}}

Following this section the thesis will be structured as follows:
\\
\\
\textbf{Chapter \ref{cha:chapter2}} will present some background information on virtualization technologies, embedded systems with ARM architecture and a brief overview of PHIDIAS.
\\
\\
\textbf{Chapter \ref{cha:chapter3}} will have an in depth look at the architecture of Xen hypervisor and the core components of its split device driver model for I/O virtualization.
\\
\\
\textbf{Chapter \ref{cha:chapter4}} will shed light on basic setup of working environment and discuss requirements and assumptions of thesis.
\\
\\
\textbf{Chapter \ref{cha:chapter5}} will feature a more thorough presentation of porting basic components of Xen split driver model while \textbf{Chapter \ref{cha:chapter6}} will describe porting of Xen network drivers to PHIDIAS.
\\
\\
\textbf{Chapter \ref{cha:chapter7}} will be about testing and evaluating the network performance on Xen and PHIDIAS. The results of latency and bandwidth tests will be presented and analyzed.
\\
\\
\textbf{Chapter \ref{cha:chapter8}} will summarize the thesis along with potential limitations and benefits, and gives an outlook on future work.
\\
\\
Appendix \ref{Appendix} features patch for the changes in source code of Linux guests and PHIDIAS related to current work in addition to some instructions for running virtual network devices and running network tests on PHIDIAS and Xen.