\chapter{Introduction \label{cha:chapter1}}

With a diverse range of applications and market opportunities of virtualization technology, it has become one of leading technologies of the future. Though it began its journey from mainframes, it soon found its place in servers as well as desktop virtualization. As the benefits of virtualization technology continues to grow, it has now entered into today's most demanding market space of embedded systems.
\\
\\
A few decades ago, embedded systems were designed to run simpler applications with special design constraints. But the present embedded systems provide more complex functionality targeting real time applications and defense mission critical systems. Running multiple applications of mixed criticality on embedded systems provided virtualization to grab another opportunity to prove its usefulness in terms of providing reliable and secure systems. 
\\
\\
The heart of virtualization technology is a software called hypervisor which allows physical hardware devices to be shared among virtual machines running on to top it. It manages memory and hardware resource allocation by running virtual machines in complete isolation. As far as I/O devices are concerned, older embedded systems consisted of fewer I/O devices targeting simpler applications. However, today's embedded systems are designed to include many different types of I/O devices for meeting the requirements of complex applications. Virtualization of these I/O devices is a core component in developing today's virtualized embedded systems. 
\\
\\
With the rapid evolution in hardware platforms, the need for developing efficient software for better hardware utilization has also increased. Multiple players are providing innovative solutions in the virtualization industry. According to 2016 Spiceworks State of IT report \cite{spiceworks}, more than 76\% of organizations are taking advantage of virtualization today, and that number will continue to grow in the future. Xen and KVM are amongst the most popular open source hypervisors. Both provide support for I/O device virtualization. However, they heavily depend upon dynamic memory and resource allocation techniques. The usage of dynamic components in their designs makes it harder for proving reliability and safety for real-time and safety critical applications.
\\
\\
PHIDIAS is a hypervisor developed by Dr.-Ing. Jan Nordholz at TU Berlin Telekom Innovation Laboratories \cite{Jan},supporting ARM, MIPS and x86. It is based on static compile time configuration of virtual machines to remove non-mandatory dynamic components from the system. The static configuration of PHIDIAS reduces dynamicity of system which results in a simpler provable code with a smaller memory footprint. However, it only provides virtualization of timer and UART devices. There is no support for other virtual I/O devices e.g. network and block devices.
\\
\\
This thesis will add support of I/O device virtualization in PHIDIAS by using Xen I/O architecture as a reference. Due to time restrictions, it will focus on running and testing virtual network device on PHIDIAS  and Xen in order to compare throughput and bandwidth. The method for supporting other I/O devices will also be explained along with benefits and proposed future research.

\section{Motivation\label{sec:moti}}
There are two main motivational factors behind this work. First is to demonstrate the fact that even though PHIDIAS is a statically configured hypervisor with a minimal set of drivers, its structure is flexible enough to add support of other virtual I/O devices. \\
\\
Second motivation is to answer questions about what choices do we have among open source hypervisors suitable for using as a reference for porting I/O virtualization to PHIDIAS, how similar would the chosen hypervisor be to nature of PHIDIAS and which hypervisor would allow using existing framework of PHIDIAS? Xen and KVM both are open source hypervisors, supporting ARM target platforms. However, Xen is a bare-metal or type-1 hypervisor similar to PHIDIAS while KVM is type-2 or hosted hypervisor. Also Xen I/O architecture can easily reuse PHIDIAS shared memory and interprocessor interrupt mechanisms for successful virtualization of I/O devices. For this thesis, Xen turned out to be the closest match for using as a reference. A detailed view of Xen will be explained in this thesis for the better understanding of its I/O architecture.

\section{Aims and Objective\label{sec:objective}}
Main goals of the current work are twofold. First is to use PHIDIAS existing memory sharing and interprocessor communication mechanisms for porting Xen I/O drivers, thus extending its functionality to accommodate virtual I/O devices. Second is to keep changes in Xen I/O drivers and PHIDIAS as minimum as possible to reuse Xen's maintenance support and to keep small footprint and static nature of PHIDIAS intact.

\section{Scope\label{sec:scope}}
The main focus of the thesis will be to find core components in Xen I/O virtualization model responsible for inter-VM communication and replace them with corresponding PHIDIAS components. The target platform will be ARM due to its wide range of applications in smart phones, laptops, digital TVs, set-top boxes, mobile, and Internet of Things `IoT' devices. 


\section{Related Work\label{sec:prework}}
There has been a lot of work on developing techniques for high-performance I/O virtualization. Over the past decade, achieving better hardware utilization by decoupling logical I/O devices from physical ones has been the focus of researchers in virtualization domain.
\\
Gordon et al. \cite{exitless} proposed an exitless paravirtual I/O model, under which guests and the hypervisor, running on distinct cores, exchange exitless notifications instead of costly exit-based notifications for increasing the performance of paravirtual I/O. Nadav et al. \cite{efficient}  developed ELVIS (Efficient and scaLable paraVirtual I/O System) for running host functionality on dedicated cores that were separate from guest cores, and thus avoiding exits on the I/O path. Shafer et al. \cite{CDNA} introduced concurrent direct network access (CDNA), a new I/O virtualization technique combining both software and hardware components that significantly reduces the overhead of network virtualization in VMMs. With CDNA, a unique context is allocated to each virtual machine on the network interface for communicating directly with the network interface through that context. In this way, tasks of traffic multiplexing, interrupt delivery, and memory protection gets divided between hardware (network interfaces) and software (virtual  machine monitor). Shinagawa, Kawai and Kono et al. \cite{bitvisor} developed a small hypervisor named BitVisor for the fast I/O access from the guest operating systems by passing-through the virtual machine monitor.
\\
\\
In addition to developing innovating software techniques for efficient virtualization of I/O subsystems and peripheral devices, researchers have also worked on adding hardware support for virtualization in I/O devices. The idea is to bypass the hypervisor by virtual machines to directly access logical interfaces implemented on actual hardware, thus producing best performance. Himanshu and Karsten \cite{selfvirt} developed a new approach of I/O virtualization by  offloading select virtualization functionality onto the device termed as self-virtualized devices. Intel's has developed Single Root I/O Virtualization (SR-IOV) technology \cite{SRIOC} that extends the physical function of the I/O port of an I/O device to a number of virtual functions directly assigned to virtual machines, hence achieving near native performance without software emulation. 
\\
\\
In virtualization industry, efforts for developing effective I/O virtualization techniques have resulted in three types of solutions i.e software-based, hardware-based and combination of both software and hardware based I/O virtualization. On one side, software-based approach, using paravirtualized drivers implemented in dedicated driver domain or as a part of hypervisor, provides isolated execution of I/O virtual device drivers and enables live migration \cite{migration} and traffic monitoring \cite{monitoring} for virtual machines. On the other side, hardware method reduces the software overhead and bypass hypervisor layer to achieve good performance. All choices have their pros and cons and selection heavily depends upon the target application.

\section{Outline\label{sec:outline}}

This section gives a brief introduction into the organization of this study. This thesis is divided into eight chapters, starting with this first Chapter \ref{cha:chapter1} on introduction. 
\\
\\
\textbf{Chapter \ref{cha:chapter2}} is called `Background' which will provide information on virtualization technologies, embedded systems with ARM architecture and a brief overview of Phidias.
\\
\\
\textbf{Chapter \ref{cha:chapter3}} will discuss the architecture of Xen hypervisor and will explain core components of its split device driver model for I/O virtualization.
\\
\\
\textbf{Chapter \ref{cha:chapter4}} will provide the environmental setup, requirements and assumptions of thesis.
\\
\\
\textbf{Chapter \ref{cha:chapter5}} will explain porting of basic components of Xen split driver model while  \textbf{Chapter \ref{cha:chapter6}} will describe porting of Xen network drivers to PHIDIAS.
\\
\\
\textbf{Chapter \ref{cha:chapter7}} will be about testing and evaluating the network performance on Xen and PHIDIAS. The results of latency and bandwidth tests will be presented and analyzed.
\\
\\
\textbf{Chapter \ref{cha:chapter8}} will summarize the thesis along with potential limitations and benefits, and gives an outlook about future work.
